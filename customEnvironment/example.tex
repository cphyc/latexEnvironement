\documentclass[]{article}
\usepackage{amsmath} % for math symbols
\usepackage{MyTheorem}
\usepackage{MyExample}
\usepackage{lipsum} % just for the example

\newthm{theorem}{Theorem}
\newcustomdef{definition}{Definition}
\newcustomexample{example}{Example}
\newcustommisc{remark}{Remark}


\begin{document}
\begin{theorem}[Law of big numbers a name (weak form)]
  Let $\left(X_n \right)_{n\in\mathbb{N}^*}$ be a sequence of independant
  random variables with the same variance $V(X)$ and the same expected
  value $E(X)$, then
  \begin{equation}
    \forall \epsilon > 0,\quad \lim_{n\rightarrow\infty} \mathbb{P} \left(\left|\frac{\sum_{i=1}^{n}X_i
}{n} - E(X)\right| \geq \epsilon  \right) = 0
  \end{equation}
\end{theorem}

\begin{remark}[Short remark]
  \lipsum[1]
\end{remark}
\begin{definition}[Short definition]
  \lipsum[1]
\end{definition}
\begin{example}[Short example]
  \lipsum[1]
\end{example}
\begin{example}[Loooooong example]
  \lipsum \lipsum
\end{example}
\end{document}

%%% Local Variables: 
%%% mode: latex
%%% TeX-master: t
%%% End: 
